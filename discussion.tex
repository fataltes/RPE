\chapter{Conclusion, Discussion, and Future Work}

When working at the scale of whole genomes, the problem of extending indexing strategies to graphs becomes very important. In addition to that, indexing more than one sample in the same data structure adds to the complexity of the problem. In this document, we presented two data structures for indexing collection of genomes, transcriptomes, or sample reads on top of colored de Bruijn graphs and compacted de Bruijn graphs. Both of these data structures make use of succinct representations along with rank and select operations. As the future direction, we are interested in exploring different applications for these two data structures and also investigating the relationship between them to see how we can merge these two into one multi-purpose data structure that can be part of different pipelines of mapping and assembly and merged with different representations of de Bruijn graph including succinct representations like BOSS and compacted de Bruijn graph.

In colored de Bruijn graph, the only information we keep per each \kmer in the indexing data structure regarding different samples (genomes, sequence read samples, etc.) is just its membership. Therefore, one future direction for indexing a colored de bruijn graph can be having different types of annotations in one compacted package in addition to just membership; annotations like position and orientation for a referenced base colored de Bruijn graph and frequency for both referenced based and assembly cases.
Moreover, it will be interesting to explore how multiple attributes could be efficiently stored simultaneously, and how potential correlations between these attributes might be exploited. For example, there may be natural extensions of similar coding schemes to the compacted de Bruijn graph, where one might also be able to take advantage of the coherence in annotation (i.e., color or count information) shared among the constituent \kmers of a contig, allowing one to store only the information where these annotations change during traversal.

Another area of interest toward improving space of colored de Bruijn graphs can be exploring other ways to use fewer colors to represent all samples instead of compressing the final color matrix. One of the approaches is to reuse the previously assigned colors in disjoint subgraphs or even non-adjacent edges. In other words, we cannot reuse the same color for a different sample for \kmers that are adjacent, but we can have colors that are globally reused. In this case we need to have a proper mapping from a color to its corresponding samples in different disjoint areas of the graph. As a de Bruijn graph is a special type of directed graphs with each node having at most $4$ incoming and $4$ outgoing edges, the chromatic number of such graph is at most $9$ (maximum degree + 1). Therefore, if we can design an efficient color-to-sample mapping, we can reduce size of the graph by not assigning one bit to each color. In this case, we are approaching the problem from a different direction than compressing the color matrix that we explained in \ref{sec:rainbowfish}.

The main advantage of a data structure like \pufferfish compared to a linear index is the ability of efficiently mapping reads to a population of genomes or individual genomes with annotated variants. Current tools that are used for alignment and mapping are either suitable for genome or transcriptome, but not both. \pufferfish fills the gap by allowing fast and accurate mapping to a collection of genomes and transcriptomes at the same time. This ability paves the way for accomplishing applications such as ``novel exon discovery" or ``RNA-seq quality control''. 
Because of the distinction between methods that map to transcriptome and those that map to genome, we lose the information that can be derived by puting both of these mappings together. One immediate outcome of having short reads mapped to both genome and transcriptome is in ``RNA-seq quality control''. If we just look at the transcriptome mapping outcome, we would just simply throw all the non-mapped reads out, ignoring the fact that not being mapped at all is a different observation than being mapped to an intron. A large fraction of reads mapping to introns is an evidence that the RNA-seq experiment failed to provide the required quality. Having reads mapped to both genome and transcriptome, we can account for such experiments' failure. Another application of dual mapping which is of a biological importance is looking for any intron retention in an RNA-seq read set. High probability of retaining introns in reads from RNA-seq experiments is known to be associated with certain desease phynotypes. By mapping the reads to just transcriptomes, we can never be aware of any intron retentions.

Another problem that can be answered using \pufferfish is finding ``structureal variations'' in a metagenomic data set or accross different samples of the same individual. One particular way to approach this problem is through genome alignment. Having a genome indexed using \pufferfish, we can map another genome to it instead of a set of short reads. In a different view, we can index multiple genomes at the same time into one data structure combining the concepts of indexing a colored de Bruijn graph and a compacted de Bruijn graph. There are a lot of small scale variations happening across two genomes that can be the result of an error in read sequencing process or not informative enough. However, the biologically interesting variations across genomes are those that happen in scales longer than the length of the read. Building \pufferfish along a collection of genomes will allow us to search for such variations. One specific type of variation can be inversions happening alongside of the genome, so that we have a one to one correspondance between \kmers of two genomes and at some location the positions start decreasing in one genome as we increase position in the other. The inversions can be identified by a data structure like \pufferfish that keeps position of a \kmer in all the references.


